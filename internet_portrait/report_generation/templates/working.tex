
\documentclass[a4paper,14pt]{extarticle} 
\usepackage[T1,T2A]{fontenc}
\usepackage[utf8]{inputenc}
\usepackage[english,russian]{babel}
\usepackage{graphicx}
\usepackage{geometry} 
\geometry{left=2.5cm}
\geometry{right=1cm}
\geometry{top=2cm}
\geometry{bottom=2cm}

\begin{document}
\begin{titlepage}
\date{}
\begin{center}
{\huge \textbf{АНАЛИТИЧЕСКИЙ ОТЧЕТ}} \\
\vspace{80mm}
{\Huge \textbf{ Интернет-портрет региона \\ {{content}} } } \\
\vspace{\baselineskip}
{\large Выполнен студией Regio Reports\par }
\vspace{\baselineskip}
\vspace{\baselineskip}
\vspace{\baselineskip}
{Москва, {{date_report_generated}}}
\end{center}
\end{titlepage}
\clearpage\maketitle
\thispagestyle{empty}
\newpage

\addcontentsline{toc}{section}{Введение}
\section*{Введение}
Необходимость проводить анализ существующей ситуации в регионе возникает постоянно. В связи с этим на передний план выходят различные методики, такие как построение социально-экономического портрета региона, или, совершенно новая, – построение интернет-портрета. 
Методика построения социокультурного портрета опирается на разработку трех основных направлений:
\begin{itemize}
\item выявление факторов, которые определяют индивидуальный облик каждого региона;
\item изучение попарных сходств и различий регионов, смежных и отдаленных;
\item сопоставлени их с факторами и тенденциями социокультурной эволюции;
\end{itemize}

Однако данный подход опирается на базу социальных исследований, проведение которых усложняется тремя обстоятельствами. Во-первых, в данный процесс вовлечено очень много людей. Во-вторых, методология составления опросников – весьма сложная процедура. В третьих, население неохотно отвечает на вопросы, что в свою очередь искажает результаты исследования. Кроме того, выборка участников опроса не всегда объективно отражает все слои населения. 

Как решение данных проблем, на передний план выходят новые способы исследования особенностей региона, такие, как построение интернет-портрета региона.

\chapter{Анализ информационного пространства}				
%\addcontentsline{toc}{chapter}{Анализ информационного пространства}

Проведем следующий анализ новостей за период от {{period_from}} до {{period_to}} в регионе {{content}}.
\begin{itemize}
	\item Количество новостей: {{number_of_news}}
	\item Количество источников: {{number_of_sources}}
	\item Наиболее важные события за данный период:
	\begin{itemize}

	
       \item {{ story.title }}
	
	\end{itemize}
\end{itemize}
\end{document}

